\begin{comment}
  * Read the Washington in detail. You can use
  the web too. I have ordered lots of crypto
  books for the Stafford library.
  
  * For chapter 1, write several classical
  ciphers. Introduce main ideas without
  using number theory. You should mention the
  concept of key and to break a cipher means
  finding the key. There are several ciphers
  I covered and there are more in the book.
  Talk about "private key" cipher.
  Include examples.

  * Chapter 2: Basic number theory.
  Talk about divisibility, gcd, Euclidean
  algorithm, mod, Fermat, Euler.
  There are other results in Washington book.
  
  * Chapter 3: RSA 
\end{comment}
  
\input{mybookpreamble.tex}
\makeindex
\renewcommand\TITLE{My CISS451 Book}
\renewcommand\AUTHOR{John Doe}
\renewcommand\SHORTAUTHOR{}

\begin{document}
\topmatter

\chapter{Introduction}

\section{The first cipher: Caesar cipher}

The following word is indexed: 
derivative\index{derivative}.

Have you heard of Julius Caesar?
He invented (or maybe his general? [CHECK])
one of
the earliest cipher -- the
\textbf{Caesar cipher}\index{Caesar cipher}.
Let me explain:

blah

\newcommand\infsum[2]{\sum_{#1 = #2}^\infty}

The most famous series you must know is the
geometric series:
\[
\infsum{n}{0} x^n = \frac{1}{1 - x}
\]
This is wrong:
\[
\infsum{k}{2} x^k = \frac{1}{1 - x}
\]

\begin{thm} (Fermat's Little Theorem)
Let $p$ be a prime.
If $\gcd(a, p) = 1$, then
\[
a^{p - 1} \equiv 1 \pmod{p}
\]
\end{thm}

\proof
Consider the function
\[
f : (\Z/p)^* \rightarrow (\Z/p)^*
\]
by
\[
f(x) = ax
\]
Then ...TODO... $f$ is a bijection.
Hence $f$ permutes $(\Z/p)^*$.
We multiply the values of $(\Z/p)^*$ to get
\[
\prod_{x \in (\Z/p)^*} x
\]
The set $(\Z/p)^*$ permuted by $f$ is
\[ 
\{ ax \mid x \in (\Z/p)^*  \}
\]
Multiplying the values of this set, we get
\[
\prod_{x \in (\Z/p)^*} (ax)
\]
... TADA.
\qed



\section{Second Section}
\lipsum[1-4]

\section{Third Section}
\lipsum[1-5]

\chapter{History from 1300 to 1500}

\section{Fourth Section }
\lipsum[1-6]

\section{Fifth Section }
\lipsum[1-7]

\section{Sixth Section }
\lipsum[1-8]

\chapter{History from 1500 to 1800}

\section{Theorem Environment}

INSIDE PARAGRAPH.
Lorem ipsum dolor sit amet, consectetur adipisicing elit, sed do eiusmod 
tempor incididunt ut labore et dolore magna aliqua. Ut enim ad minim veniam, 
quis nostrud exercitation ullamco laboris nisi ut aliquip ex ea commodo 
\begin{thm}
Lorem ipsum dolor sit amet, consectetur adipisicing elit, sed do eiusmod 
tempor incididunt ut labore et dolore magna aliqua. Ut enim ad minim veniam, 
quis nostrud exercitation ullamco laboris nisi ut aliquip ex ea commodo 
\[
x = 42
\]
Lorem ipsum dolor sit amet, consectetur adipisicing elit, sed do eiusmod 
\begin{align*}
x &= y \\
  &= z
\end{align*}
tempor incididunt ut labore et dolore magna aliqua. Ut enim ad minim veniam, 
quis nostrud exercitation ullamco laboris nisi ut aliquip ex ea commodo 
\end{thm}
\noindent 
consequat. Duis aute irure dolor in reprehenderit in voluptate velit esse 
cillum dolore eu fugiat nulla pariatur. Excepteur sint occaecat cupidatat non 
proident, sunt in culpa qui officia deserunt mollit anim id est laborum.

TOP OF PARAGRAPH.
\begin{thm}
\lipsum[10]
\end{thm}
\noindent
\lipsum[1]

BOTTOM OF PARAGRAPH.
\lipsum[1]
\begin{thm}
\lipsum[10]
\end{thm}


\chapter{History from 1800 to 2000}

\section{Verbatim}
TEST VERBATIM WITHIN PARAGRAPH.
Lorem ipsum dolor sit amet, consectetur adipisicing elit, sed do eiusmod 
tempor incididunt ut labore et dolore magna aliqua. Ut enim ad minim veniam, 
quis nostrud exercitation ullamco laboris nisi ut aliquip ex ea commodo 
\begin{Verbatim}[frame=single, numbers=left]
This is verbatim.
\end{Verbatim}
consequat. Duis aute irure dolor in reprehenderit in voluptate velit esse 
cillum dolore eu fugiat nulla pariatur. Excepteur sint occaecat cupidatat non 
proident, sunt in culpa qui officia deserunt mollit anim id est laborum.

TEST VERBATIM AT BOTTOM OF PARAGRAPH
\lipsum[1]
\begin{Verbatim}[frame=single, numbers=left]
This is verbatim.
\end{Verbatim}

\section{Test Enumerate}
TEST WITHIN PARAGRAPH.
\lipsum[1]
\begin{enumerate}
\item ITEM 1
\item ITEM 2
\item ITEM 3
\item ITEM 4
\end{enumerate}
\lipsum[2]


TEST AT BOTTOM OF PARAGRAPH.
\lipsum[1]
\begin{enumerate}
\item ITEM 1
\item ITEM 2
\item ITEM 3
\item ITEM 4
\end{enumerate}

\lipsum[1]

\lipsum[1]


\section{A New Section}
\lipsum[1-2]
\begin{samepage}
\begin{longtable}{|c|c|c|} \hline
a & b & c \\ \hline
a & b & c \\
a & b & c \\
a & b & c \\
a & b & c \\ \hline
\end{longtable}
\end{samepage}

\section{Yet Another New Section}
\lipsum[3-4]

\section{And Yet Another New Section}
\lipsum[3-4]

\section{And Yet Another New Section}
\lipsum[3-4]

\section{And Yet Another New Section}
\lipsum[3-4]



\cleardoublepage
\addcontentsline{toc}{chapter}{Bibliography}
\begin{thebibliography}{999}
\input{bibliography.tex}
\end{thebibliography}


\cleardoublepage
\addcontentsline{toc}{chapter}{Index}
\printindex


\end{document}
